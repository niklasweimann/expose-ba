\documentclass[fontsize=11pt, paper=a4, parskip=half]{scrartcl}
\pagestyle{plain}

\usepackage[T1]{fontenc}
\usepackage[utf8]{inputenc}
\usepackage[ngerman]{babel}
\usepackage{url}
\usepackage{hyperref}
\usepackage{lipsum}

\setkomafont{disposition}{\normalcolor\bfseries}

% In den folgenden Zeilen {...} jeweils durch die tatsächliche personen-/exposébezogene Angaben ergänzen
\title{
	\{TITEL DES EXPOSÉS\}
}

\subtitle{Exposé zur 
	Bachelorarbeit
}

\author{
	Niklas Weimann
	\\
	\texttt{niklas.weimann@student.uni-siegen.de}
	\\ \\
	{1352285}
}

\begin{document}

\maketitle

\section{Motivation}
Für das Verständnis von Verschlüsselungen ist es wichtig, dass bei Inhalten eine möglichst geringe intellektuelle Hürde gesetzt wird, um den Lernenden einen einfachen Einstieg in ein komplexes Themengebiet ermöglichen zu können. Dazu ist es hilfreich, wenn das weltweit eingesetzte E-Learningtool CrypTool2 möglichst viele einfache Verfahren unterstützt, um die Grundprinzipien von kryptographischen Verfahren vielfältig und anschaulich darstellen zu können. Durch die ansprechende Visualisierung kann der individuelle Lernprozess des Anwenders unterstützt werden.

Neben dem Fokus Lernende bei ihrem Lernprozess zu unterstützten versucht CrypTool2 ebenso eine Anlaufstelle für professionelle Anwender zu bieten, die sich beispielsweise auf die Analyse von historischen Ereignissen spezialisiert haben. Nicht selten kommt es vor, dass historische Texte durch ein kryptographisches Verfahren geschützt wurden. Diese zumeist recht einfachen kryptographischen Verfahren lassen sich mittels moderner Computer und Analysetechniken jedoch leicht brechen. Hierzu ist es für diese Anwendergruppe nützlich, wenn CrypTool2 für verschiedenste klassische Verfahren eine geeignete Kryptoanalyse anbieten kann. Diese Analysetechniken können dann entweder allgemein für verschiedene kryptographische Verfahren angewendet werden oder speziell auf ein Verfahren ausgerichtet sein.

\section{Problemstellung}




\section{Zielsetzung}
In dieser Arbeit sollen verschiedene klassische Verschlüsselungsverfahren als Plugins für das E-Learning Tool CrypTool2 entwickelt werden. Zusätzlich zur Ver-/ und Entschlüsselung soll ebenso für alle Verfahren eine geeignete Krytographischeanalyse implementiert werden, soweit dies noch nicht durch bereits vorhandene Komponenten mit CrypTool2 möglich ist. 

\section{Vorgehensweise}
Zunächst sollen in dieser Arbeit die einzelnen Verfahren genauer erklärt werden, sowie ihre Angreifbarkeit hinsichtlich cryptographischer Analyse dargestellt werden. Basierend auf diesen Erkenntnissen werden Komponenten für das E-Learningtool CrypTool2 entwickelt, die sowohl die Ver-, als auch Entschlüsselung der entsprechenden Verfahren unterstützen.\citation{Erl2005}

\pagebreak

\bibliographystyle{alphadin}
\bibliography{literature}
\end{document}
